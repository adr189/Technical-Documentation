\documentclass[11pt,a4paper]{report}
\usepackage[portuguese]{babel}
\usepackage[utf8]{inputenc}
\usepackage{graphicx}
\usepackage{enumerate}
\usepackage{hyperref}
\usepackage{multicol}
\usepackage{titlepic}
\usepackage{color}
\usepackage{geometry}
\usepackage{ragged2e}
\usepackage{adjustbox}
\usepackage{tabularx}
\usepackage[table, dvipsnames]{xcolor}
\author{Alexandre Dinis Bravo Leal Rodrigues \\fc54472 - TP 11\\}
\title{Produção de Documentos Técnicos}
\date{Licenciatura em Engenharia Informática\\fc54472@alunos.fc.ul.pt}
\titlepic{\includegraphics[scale=1]{fcul}}
\makeindex
\geometry{a4paper,top=2 cm,bottom=2 cm}

\begin{document}
\maketitle
\pagenumbering{Roman}
\thispagestyle{empty}
\tableofcontents
\listoffigures
\listoftables

\chapter{Introdução}
\pagenumbering{arabic}
\setcounter{page}{1}
\paragraph{}
As pessoas, em geral, têm uma boa noção sobre quais são as práticas necessárias para ter uma vida saudável: alimentação equilibrada, prática de exercício físico, beber bastante água, ter uma boa noite de sono, evitar a ingestão de álcool, uso de cigarros e stress. Mas o que não sabemos é que existem muitos outros hábitos que fazem mal, mas que parecem inofensivos. Promover mudanças de hábitos na rotina é o primeiro passo rumo a um estilo de vida saudável.
Hábitos pessoais a corrigir:
\begin{table} [h!]
\begin{tabularx}{10 cm}{X  X }
Roer unhas & Má postura\\
Levar a vida muito a sério & Desarrumação \\ 
Pessimismo & Falta de concentração\\ 
\end{tabularx}
\end{table}
\chapter{Caso de estudo}
\section{Caso de estudo A}
\paragraph{}
No seguinte caso de estudo vamos abordar o assunto do lixo marinho, mais precisamente, os plásticos existentes no mar. A empresa Zoreg pretende reciclar plástico e fabricar uma nova embalagem reciclada e resistente. Para isso, criou-se um caso de estudo em que a produtividade será de 62 com 40 funcionários.

\begin{table} [h]
\begin{center}
\caption{\bf Tabela com 40 funcionários}
\begin{adjustbox}{max width=0.9\textwidth,max totalheight=\textheight,keepaspectratio}
\begin{tabular} {|c|c|c|c|c|c|c|c|}
\hline
Dia	& Nº Aleatório & Nº Encomendas recebidas &	Nº Encomendas a Produzir &	Capacidade Produção diária &	Nº Encomendas Produzidas &	Nº Encomendas em Atraso &	Percentagem Ocupação Mão-de-obra \\ 
\hline 
1 &	0,526 &	32500 &	32500 &	7440 &	7440 &	25060 &	100,00\% \\ 
\hline 
2 &	0,406 &	30000 &	55060 &	7440 &	7440 &	47620 &	100,00\%\\ 
\hline 
3 &	0,788 &	35000 &	82620 &	7440 &	7440 &	75180 &	100,00\%\\ 
\hline 
4 &	0,439 &	30000 &	105180 &	7440 &	7440 &	97740 &	100,00\%\\ 
\hline 
5 &	0,453 &	32500 &	130240 &	7440 &	7440 &	122800 &	100,00\%\\ 
\hline 
6 &	0,093 &	25000 &	147800 &	7440 &	7440 &	140360 &	100,00\%\\ 
\hline 
7 &	0,958 &	40000 &	180360 &	7440 &	7440 &	172920 &	100,00\%\\ 
\hline 
8 &	0,707 &	35000 &	207920 &	7440 &	7440 &	200480 &	100,00\%\\ 
\hline 
9 &	0,256 &	30000 &	230480 &	7440 &	7440 &	223040 &	100,00\%\\ 
\hline 
10 &	0,950 &	40000 &	263040 &	7440 &	7440 &	255600 &	100,00\%\\ 
\hline 
11 &	0,294 &	30000 &	285600 &	7440 &	7440 &	278160 &	100,00\%\\ 
\hline 
12 &	0,115 &	27500 &	305660 &	7440 &	7440 &	298220 &	100,00\%\\ 
\hline 
13 &	0,321 &	30000 &	328220 &	7440 &	7440 &	320780 &	100,00\%\\ 
\hline 
14 &	0,144 &	27500 &	348280 &	7440 &	7440 &	340840 &	100,00\%\\ 
\hline 
15 &	0,602 &	32500 &	373340 &	7440 &	7440 &	365900 &	100,00\%\\ 
\hline 
16 &	0,122 &	27500 &	393400 &	7440 &	7440 &	385960 &	100,00\%\\ 
\hline 
17 &	0,412 &	30000 &	415960 &	7440 &	7440 &	408520 &	100,00\%\\ 
\hline 
18 &	0,342 &	30000 &	438520 &	7440 &	7440 &	431080 &	100,00\%\\ 
\hline 
19 &	0,953 &	40000 &	471080 &	7440 &	7440 &	463640 &	100,00\%\\ 
\hline 
20 &	0,739 &	35000 &	498640 &	7440 &	7440 &	491200 &	100,00\%\\ 
\hline 
21 &	0,692 &	32500 &	523700 &	7440 &	7440 &	516260 &	100,00\%\\ 
\hline 
22 &	0,557 &	32500 &	548760 &	7440 &	7440 &	541320 &	100,00\%\\ 
\hline 
23 &	0,443 &	30000 &	571320 &	7440 &	7440 &	563880 &	100,00\%\\ 
\hline 
24 &	0,257 &	30000 &	593880 &	7440 &	7440 &	586440 &	100,00\%\\ 
\hline 
25 &	0,470 &	32500 &	618940 &	7440 &	7440 &	611500 &	100,00\%\\ 
\hline 
26 &	0,674 &	32500 &	644000 &	7440 &	7440 &	636560 &	100,00\%\\ 
\hline 
27 &	0,281 &	30000 &	666560 &	7440 &	7440 &	659120 &	100,00\%\\ 
\hline 
28 &	0,896 &	40000 &	699120 &	7440 &	7440 &	691680 &	100,00\%\\ 
\hline 
29 &	0,889 &	40000 &	731680 &	7440 &	7440 &	724240 &	100,00\%\\ 
\hline 
30 &	0,128 &	27500 &	751740 &	7440 &	7440 &	744300 &	100,00\%\\ 
\hline 
31 &	0,871 &	35000 &	779300 &	7440&	7440 &	771860 &	100,00\%\\ 
\hline 
\end{tabular}
\end{adjustbox}
\end{center}
\end{table}
\begin{figure} [h]
\begin{center}
\caption{\bf Percentagem de mão de obra}
\includegraphics[scale=0.75]{grafico1}
\end{center}
\label{figura1}
\end{figure}

\section{Caso de estudo B}
\paragraph{}
No seguinte caso de estudo vamos abordar o assunto do lixo marinho, mais precisamente, os plásticos existentes no mar. A empresa Zoreg pretende reciclar plástico e fabricar uma nova embalagem reciclada e resistente. Para isso, criou-se um caso de estudo em que a produtividade será de 62 com 180 funcionários.
\begin{table} [h]
\begin{center}
\caption{\bf Tabela com 180 funcionários}
\begin{adjustbox}{width=1\textwidth,max totalheight=\textheight,keepaspectratio}
\begin{tabular} {|c|c|c|c|c|c|c|c|}
\hline 
Dia	& Nº Aleatório & Nº Encomendas recebidas &	Nº Encomendas a Produzir &	Capacidade Produção diária &	Nº Encomendas Produzidas &	Nº Encomendas em Atraso &	Percentagem Ocupação Mão-de-obra \\ 
\hline 
1 &	0,821 &	35000 &	35000 &	33480 &	33480 &	1520 &	100,00\%\\ \hline 
2 &	0,305 &	30000 &	31520 &	33480 &	31520 &	0 &	94,15\%\\ \hline
3 &	0,464 &	32500 &	32500 &	33480 &	32500 &	0 &	97,07\%\\ \hline
4 &	0,131 &	27500 &	27500 &	33480 &	27500 &	0 &	82,14\%\\ \hline
5 &	0,049 &	25000 &	25000 &	33480 &	25000 &	0 &	74,67\%\\ \hline
6 &	0,516 &	32500 &	32500 &	33480 &	32500 &	0 &	97,07\%\\ \hline
7 &	0,631 &	32500 &	32500 &	33480 &	32500 &	0 &	97,07\%\\ \hline
8 &	0,265 &	30000 &	30000 &	33480 &	30000 &	0 &	89,61\%\\ \hline
9 &	0,124 &	27500 &	27500 &	33480 &	27500 &	0 &	82,14\%\\ \hline
10 &	0,765 &	35000 &	35000 &	33480 &	33480 &	1520 &	100,00\%\\ \hline
11 &	0,41 &	30000 &	31520 &	33480 &	31520 &	0 &	94,15\%\\ \hline
12 &	0,776 &	35000 &	35000 &	33480 &	33480 &	1520 &	100,00\%\\ \hline
13 &	0,103 &	27500 &	29020 &	33480 &	29020 &	0 &	86,68\%\\ \hline
14 &	0,881 &	40000 &	40000 &	33480 &	33480 &	6520 &	100,00\%\\ \hline
15 &	0,855 &	35000 &	41520 &	33480 &	33480 &	8040 &	100,00\%\\ \hline
16 &	0,92 &	40000 &	48040 &	33480 &	33480 &	14560 &	100,00\%\\ \hline
17 &	0,655 &	32500 &	47060 &	33480 &	33480 &	13580 &	100,00\%\\ \hline
18 &	0,716 &	35000 &	48580 &	33480 &	33480 &	15100 &	100,00\%\\ \hline
19 &	0,099 &	25000 &	40100 &	33480 &	33480 &	6620 &	100,00\%\\ \hline
20 &	0,777 &	35000 &	41620 &	33480 &	33480 &	8140 &	100,00\%\\ \hline
21 &	0,57 &	32500 &	40640 &	33480 &	33480 &	7160 &	100,00\%\\ \hline
22 &	0,495 &	32500 &	39660 &	33480 &	33480 &	6180 &	100,00\%\\ \hline
23 &	0,991 &	40000 &	46180 &	33480 &	33480 &	12700 &	100,00\%\\ \hline
24 &	0,694 &	32500 &	45200 &	33480 &	33480 &	11720 &	100,00\%\\ \hline
25 &	0,13 &	27500 &	39220 &	33480 &	33480 &	5740 &	100,00\%\\ \hline
26 &	0,141 &	27500 &	33240 &	33480 &	33240 &	0 &	99,28\%\\ \hline
27 &	0,221 &	27500 &	27500 &	33480 &	27500 &	0 &	82,14\%\\ \hline
28 &	0,371 &	30000 &	30000 &	33480 &	30000 &	0 &	89,61\%\\ \hline
29 &	0,905 &	40000 &	40000 &	33480 &	33480 &	6520 &	100,00\%\\ \hline
30 &	0,555 &	32500 &	39020 &	33480 &	33480 &	5540 &	100,00\%\\ \hline
31 &	0,15 &	27500 &	33040 &	33480 &	33040 &	0 &	98,69\%\\
\hline 
\end{tabular}
\end{adjustbox}
\end{center}
\end{table}
\begin{figure} [h]
\begin{center}
\caption{\bf Percentagem de mão de obra}
\includegraphics[scale=1]{grafico2}
\end{center}
\label{figura1}
\end{figure}
\chapter{Conclusão}
\paragraph{}
Tendo em conta as tabelas e gráficos apresentados, conclui-se que o número de funcionários escolhido é o mais acertado, devido a que, graças à influência do número de aluno (que acrescenta a soma dos seus algarismos a 40 para chegar à quantidade de kg de lixo apanhados nas praias portuguesas), a produtividade e a perçentagem de ocupação de mão-de-obra mantém-se o mais proveitosas possíveis.  
\begin{thebibliography}{}
\bibitem {EstadoZen}
7 pequenos hábitos que podem mudar a sua vida by EstadoZen.\par
[https://estadozen.com/artigos/7-pequenos-habitos-que-podem-mudar-sua-vida]
\bibitem {eCycle}
Oito mudanças de hábito para ter mais saúde by eCycle.\par
[https://www.ecycle.com.br/3371-mudancas-de-habitos]
\end{thebibliography}
\end{document}